\section{Introduction}


The conventions and notes for generating basis function for the five-parameter
grain boundary space are listed.

\begin{itemize}
\item \textcolor{red}{Should we update} \verb|symm_orders| \textcolor{red}{based
    on grain exchange symmetry and Laue group symmetry?}.
  \begin{enumerate}
  \item Before
  \end{enumerate}
\item In the function, \verb|null_mat_ab|, \textbf{In line 25, why am I
    dividing} \verb|C_val| by \verb|sqrt(2*e+1)| and \textbf{In line 29, how did
    I define the normalizing constant}. \textcolor{red}{\textbf{Check with
      Jeremy!!}}
\end{itemize}

\clearpage
\newpage

\subsection{Computing the Null Boundary Singularity Matrix}


% \lstinputlisting[caption = {Code for computing Null Boundary Operation for fixed $(a,b)$}]{null_mat_ab.m}
% \newpage

\subsubsection{Description of \texttt{null\_mat\_ab.m}}

The matrix \verb|mat_ab| represents the null-boundary operation for fixed values
of $(a,b)$. The operation for null-boundary singularity is:

\begin{equation*}
\sum_{a b} \sum_{\alpha \beta \gamma} \bigg[ \Pi_{a b} C^{e -\epsilon}_{a \alpha b \beta} C^{e 0}_{a \gamma b -\gamma} \bigg] c^{a b}_{\alpha \beta \gamma} = \sqrt{2 \pi^3} f_0 \delta_{e 0} \delta_{\epsilon 0}
\end{equation*}

For fixed values of $(a,b)$, the number of coefficients $c^{a b}_{\alpha \beta
  \gamma}$ is equal to $n_{col} = (2a+1)(2b+1)(2c+1)$, where $c = \min (a,b)$.
Therefore, the number of columns of \verb|mat_ab| is equal to $n_{col}$. The
number of rows of of \verb|mat_ab| is computed as follows:

\begin{itemize}

\item The values of $e$ are constrained to lie between $|a-b|$ and $(a+b)$, i.e.
  $|a-b| \leq e \leq (a+b)$. These indices are stored in the variable
  \verb|e_range|.
\item For a given value of $e$, $\epsilon$ ranges from $-e$ to $e$. That is, $-e
  \leq \epsilon \leq e$. This is provided by the variable \verb|eps_range|.
\item Therefore, the total number of rows is given by

  \begin{equation*}
    n_{rows} = \sum_{e = |a-b|}^{a+b} \sum_{-e}^{e} \epsilon
  \end{equation*}

  This is given by the value in the expression \verb|sum(2*e_range+1)| in line
  number 9.
    
\item The Clebsch-Gordan coefficient is given by the function
  \verb|clebsch_gordan(j1, j2, j, m)| for possible values of
  $C^{j,m}_{j1,m1;j2,m2}$. The function returns the coefficients \verb|C| and
  the values of \verb|m1| and \verb|m2| for which $C^{j,m}_{j1,m1;j2,m2} \neq
  0$.
    
\item This function, \verb|clebsch_gordan(j1, j2, j, m)|, is used to determine
  the indices containing non-zero values for a given row, \verb|r_ct|. This is
  accomplished using the lines \textbf{20} and \textbf{23}.
    
  \begin{itemize}
  \item In line \textbf{20}, the indices containing non-zero values for any
    fixed value of $\gamma$ are computed.
  \item In line \textbf{23}, the non-zero values are repeated for every value of
    $\gamma$. The corresponding indices are computed.
  \end{itemize}
    
\item \textbf{In line 25, why am I dividing} \verb|C_val| by \verb|sqrt(2*e+1)|
  and \textbf{In line 29, how did I define the normalizing constant}.
  \textcolor{red}{\textbf{Check with Jeremy!!}}

\end{itemize}

\clearpage
\newpage

% \lstinputlisting[caption = {Code for computing Null Boundary Operation for a range of $(a,b)$ values}]{generate_gb_null.m}
% \pagebreak

\subsubsection{Description of \texttt{generate\_gb\_null.m}}

This code computes the Null Boundary Operation for a range of $(a,b)$ values
provided in the \verb|symm_orders| array. The matrix \verb|null_mat| gives the
operation for null-boundary singularity. The function \verb|null_mat_ab| gives
the operation for a fixed $(a,b)$. However, we are interested in the operation
for all possible values of $(a,b)$ such that $\max(a+b) \leq N$ (N is denoted by
the variable \verb|Nmax|).

\begin{itemize}
\item The array \verb|symm_orders| contains all the possible values of $(a,b)$
  such that symmetrized basis function (symmetrized using crystal rotation point
  group symmetries) exist.
\item The variable \verb|nsymm| gives the total number of basis functions
  $M^{a,b}_{\alpha, \beta, \gamma}$ for $(a,b)$ values listed in
  \verb|symm_orders|.
\item The for-loop in lines \textbf{17} to \textbf{25} gives the appropriate
  row-indices (\verb|row_inds|) and appropriate col-indices (\verb|col_inds|),
  where the matrix operation computed using \verb|null_mat_ab| will be added to
  the complete \verb|null_mat|.
\item Using the row- and column-indices computed in the previous for-loop, the
  for loop in lines \textbf{29} to \textbf{36}, calculates the
  \verb|null_mat_ab| for each combination of $(a,b)$ listed in
  \verb|symm_orders| and add this to the matrix \verb|null_mat|.
\end{itemize}

\clearpage
\newpage

\begin{equation}
  \label{eq:mbp}
\tensor[_{\gamma}]{M}{^a_{\alpha}^b_{\beta}}
\end{equation}
% \lstinputlisting[caption = {Code for computing $Y_{\pi}$ operation a given $(a,b)$ value.}]{generate_ypi_left_ab.m}